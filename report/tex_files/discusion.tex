\newpage

\section{Discusión}

\hfill

A continuación se procede a discutir los resultados que hemos obtenido de las distintas herramientas

\hfill

Tras investigar las distintas interacciones proteína-proteína (véase la figura \ref{fig:string1}) y observar las enfermedades relacionadas con la discalculia en HPO y los distintos artículos científicos citados anteriormente, suponemos la estrecha relación de la discapacidad con un mal funcionamiento del sistema nervioso. En concreto se podría teorizar que existe una relación con las enfermedades de demencia del complejo frontotemporal del cerebro y la esclerosis lateral amiotrófica, pues estas palabras claves aparecen en la mitad de las enfermedades relacionadas con el fenotipo.

\hfill

La demencia de lóbulo frontotemporal, según el estudio \cite{FrontotemoralDementia} es un síndrome clínica y patológicamente heterogéneo, caracterizado por una disminución progresiva del comportamiento o del lenguaje asociado con la degeneración de los lóbulos frontal y temporal anterior. Podemos teorizar que además de con una disminución de las habilidades lingüísticas, esta demencia guarda relación con una disminución de las habilidades aritméticas relacionadas con la discalculia o al menos, los genes implicados en la demencia frontotemporal también son de gran relevancia en la aparición de la discalculia

\hfill

La esclerosis lateral amiotrófica es una enfermedad neurodegenerativa cuyas  pruebas clínicas incluyen signos de daño de las neuronas motoras superior e inferior tanto en las extremidades como en la musculatura bulbar, y en algunos pacientes hay deterioro cognitivo frontotemporal \cite{EsclerosisLateraAmiotrófica}. De nuevo encontramos una relación con el deterioro del lóbulo frontotemporal, en una enfermedad que estaba relacionada a través de muchos de los genes implicados con nuestro fenotipo a estudiar; cada vez se refuerza mas la hipótesis de que nuestro fenotipo esta relacionado con el deterioro del lóbulo frontotemporal

\hfill

A raíz de los artículos recomendados por la base de datos STRING \cite{Walterfang2014,frontotemporal}, suponemos una relación de estos efectos negativos con una mala conexión entre los hemisferios cerebrales, recordemos que el cerebro delega algunas funciones clave como en este caso puede ser la realización de operaciones matemáticas o el reconocimiento numérico y las interconecta a través del Cuerpo Calloso \cite{CorpusCallosum}. Por tanto, un mal funcionamiento de este elemento del sistema puede llevar a una incapacidad de conexión de las funciones de los distintos hemisferios.

\hfill

Observando algunas de las componentes celulares mas relacionas con nuestro conjunto de genes originales (obtenidas del primer enriquecimiento),
encontramos el termino GO “GO:0036464”, este mismo se refiere a la ribonucleoproteina citoplasmática, este tipo de proteína guarda una estrecha relación con los cuerpos GW (cuerpos citoplasmáticos ricos en glicina y triptófano) y tienen funciones altamente especializadas, entre esas funciones se encuentran el transporte neuronal \cite{CytoplasmicRibo}, lo que podría reforzar la hipótesis de que una mala conexión entre ambos hemisferios del cerebro puede originar la discalculia.

\newpage

El segundo termino GO con mas significancia estadística del enriquecimiento a la comunidad de genes original, GO:0035770, también se refiere a las ribonucleproteinas, por lo que se puede intuir que los genes que están implicados con nuestro fenotipo también lo están, en menor medida, con las ribonucleoproteinas que anteriormente hemos supuesto que podrían tener relación con el transporte neuronal \cite{CytoplasmicRibo} que reforzaba nuestra hipótesis del fallo de conexión de los hemisferios cerebrales.

\hfill 

Tras un estudio de los análisis de enriquecimiento de las componentes celulares de las tres comunidades que hemos enriquecido, se destaca la presencia del termino GO:0035578, que se corresponde con la componente celular “lumen de gránulo azurófilo”. Parece ser que esta componente celular tienen un papel en la proteína aumentadora de la permeabilidad \cite{Azurophil}, con lo cual un fallo en la formación de esta componente podría llevar a fallos de conexión entre componentes del sistema nervioso.

\hfill

Además se encontró que este tipo de gránulos azurófilos guardan relación los promielocitos neutrófilos de la médula ósea \cite{Azurophil}, lo que reforzaría la relación de los genes vinculados con la discalculia con la enfermedad  esclerosis lateral amiotrófica anteriormente discutida.