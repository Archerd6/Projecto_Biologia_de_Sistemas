\section{Discusión}

A continuación se procede a ...

\hfill

A raíz de los artículos recomendados por la base de datos STRING \cite{Walterfang2014,frontotemporal}, suponemos una relación de estos efectos negativos con una mala conexión entre los hemisferios cerebrales, recordemos que el cerebro delega algunas funciones clave como en este caso puede ser la realización de operaciones matemáticas o el reconocimiento numérico y las interconecta a través del Cuerpo Calloso \cite{CorpusCallosum}. Por tanto, un mal funcionamiento de este elemento del sistema puede llevar a una incapacidad de conexión de las funciones de los distintos hemisferios.

\hfill

Tras investigar las distintas interacciones proteína-proteína (véase la figura \ref{fig:string1}) y observar las enfermedades relacionadas con la discalculia en HPO y los distintos artículos científicos citados anteriormente,suponemos la estrecha relación de la discapacidad con un mal funcionamiento del sistema nervioso. En concreto se podría teorizar que existe una relación con las enfermedades de demencia del complejo frontotemporal del cerebro y la esclerosis lateral amiotrófica, pues estas palabras claves aparecen en la mitad de las enfermedades relacionadas con el fenotipo.




\newpage