\section{Conclusiones}

\hfill

Remitiéndonos a los resultados y discusiones que hemos obtenido de nuestra investigación sobre la discalculia, podemos concluir que:

\hfill

Los resultados sugieren una posible, en pacientes diagnosticados de discalculia, correlación entre la demencia del lóbulo frontotemporal y el mal funcionamiento entre la interconexión de los hemisferios cerebrales que desemboca en un problema con las capacidades aritmético-lógicas. Un análisis de esa correlación a través de un estadístico en este par de características permitiría una verificación adecuada de esta idea en pacientes con discalculia.

\hfill

Además los resultados siguieren una relación entre los genes responsables de la \textit{esclerosis lateral amiotrófica} y la \textit{discalculia},que a su vez se relaciona con la \textit{demencia del lóbulo frontotemporal}; basándonos en la relación entre este tipo de esclerosis y la demencia del lóbulo frontotemporal, teorizamos que estos tres elementos (demencia del lóbulo frontotemporal, esclerosis lateral amiotrófica y discalculia) forman parte de una red cuya causa-efecto relacionan mecanismos de origen similares, y estos pueden llevar a generar en ciertas circunstancias, una de las patologías u otra.

\hfill

Finalmente, y la vista de nuestras conclusiones, sería interesante realizar estudios en pacientes con discalculia a los que, en un determinado tiempo de estudio, se les proporcionen medicamentos que faciliten la interconexión de los hemisferios cerebrales y el correcto funcionamiento nervioso del lóbulo frontotemporal; y observar si su capacidad aritmético lógica varia con el uso de dichos medicamentos.


