\section{Conclusiones}

\hfill

Remitiéndonos a los resultados y discusiones que hemos obtenido de nuestra investigación sobre la discalculia, podemos concluir que:

\hfill

Parece existir, en pacientes diagnosticados de discalculia, una correlación entre la demencia del lóbulo frontotemporal y el mal funcionamiento entre la interconexión de los hemisferios cerebrales que desemboca en un problema con las capacidades aritmético-lógicas.

\hfill

Además parece existir una relación entre los genes responsables de la esclerosis lateral amiotrófica y los responsables de la discalculia, basándonos en la relación entre este tipo e esclerosis y la demencia del lóbulo frontotemporal, teorizamos que estos tres elementos (demencia del lóbulo frontotemporal, esclerosis lateral amiotrófica y discalculia)  forman una red cuya causa-efecto relacionan mecanismos similares que generan, en ciertas circunstancias, uno de los elementos u otro.

