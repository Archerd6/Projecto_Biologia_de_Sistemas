\section{Materiales y métodos}

Se proceden a explicar los distintos recursos que se van a usar para obtener resultados a base de experimentos y procedimientos:

\subsection{Métodos}

En esta parte exponemos los distintos métodos con los que hemos llevado a cabo los experimentos.

\subsubsection{Análisis por comunidades}

En los análisis de comunidades \cite{Comun_Analisis} se busca la detección y el análisis de grupos. Esto es de enorme aplicabilidad en diferentes dominios: La detección de comunidades es un problema que no tiene un objetivos o una solución global esperada, ya que la naturaleza de las comunidades no se conoce de antemano. El problema se vuelve aún más complicado debido al hecho de que las comunidades emergen en la red en varias formas, como disjuntas, superpuestas y jerárquicas\cite{ComAnalisisStruct}.

\hfill

A lo largo de la historia se han propuesto varias heurísticas\cite{Comun_Analisis_2} para abordar estos desafíos, dependiendo de la aplicación en cuestión. Todas estas heurísticas se han materializado en forma de nuevas métricas, que en la mayoría de los casos se utilizan como funciones de optimización para detectar la estructura de la comunidad, o proporcionan una indicación de la bondad de las comunidades detectadas durante la evaluación.


\subsection{Materiales}

En esta sección serán explicados los recursos web y las librerías de R (Añadir cita R) utilizadas para los experimentos

\subsubsection{Human Phenotype Ontology}

HPO (Human Phenotype Ontology)\cite{HPO_paper} es un vocabulario estandarizado de anomalías fenotípicas en enfermedades humanas que utiliza un fenotipado detallado/preciso para poder ser usado a nivel computacional.

\hfill

\subsubsection{String}

STRING \cite{String_paper} es una base de datos de interacciones proteína-proteína conocidas y previstas. Las interacciones incluyen asociaciones directas (físicas) e indirectas (funcionales); se derivan de la predicción computacional, de la transferencia de conocimiento entre organismos y de interacciones agregadas de otras bases de datos (primarias).

\subsubsection{igraph}

Los objetivos de igraph son proporcionar un conjunto de tipos de datos y funciones para la implementación sencilla de algoritmos gráficos, el manejo rápido de gráficos grandes y permitir la creación rápida de prototipos a través de alta lenguajes de nivel como R

\subsubsection{Linkcomm}

Las comunidades de enlaces revelan la estructura anidada y superpuesta en las redes y descubren los nodos clave que forman conexiones con múltiples comunidades. linkcomm \cite{Linkcomm_paper} proporciona herramientas para generar, visualizar y analizar comunidades de enlaces en redes de tamaño y tipo arbitrario.

\subsubsection{ClusterProfiler}





\hfill




\newpage