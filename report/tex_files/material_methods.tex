\section{Materiales y métodos}

Se proceden a explicar los distintos recursos que se van a usar para obtener resultados a base de experimentos y procedimientos:

\subsection{Métodos}

En esta parte exponemos los distintos métodos con los que hemos llevado a cabo los experimentos.

\subsubsection{Análisis por comunidades}

En los análisis de comunidades \cite{Comun_Analisis} se busca la detección y el análisis de grupos. Esto es de enorme aplicabilidad en diferentes dominios: La detección de comunidades es un problema que no tiene un objetivos o una solución global esperada, ya que la naturaleza de las comunidades no se conoce de antemano. El problema se vuelve aún más complicado debido al hecho de que las comunidades emergen en la red en varias formas, como disjuntas, superpuestas y jerárquicas \cite{ComAnalisisStruct}.

\hfill

A lo largo de la historia se han propuesto varias heurísticas \cite{Comun_Analisis_2} para abordar estos desafíos, dependiendo de la aplicación en cuestión. Todas estas heurísticas se han materializado en forma de nuevas métricas, que en la mayoría de los casos se utilizan como funciones de optimización para detectar la estructura de la comunidad, o proporcionan una indicación de la bondad de las comunidades detectadas durante la evaluación.

\hfill

\subsubsection{Enriquecimiento funcional}

Se han desarrollado muchos métodos para inferir y razonar acerca de las redes de interacción molecular. Uno de ellos es el análisis funcional de genes \cite{enriquecimiento}; en éste se anota y se analiza estadísticamente una de listas de genes utilizando métodos estadísticos para identificar anotaciones funcionales sobre los genes analizados para ver si están significativamente relacionados.

\hfill

Muchas enfermedades humanas están relacionadas \cite{Enriquecimiento2} entre sí por causas comunes o incluso por patologías parecidas. El conocimiento de estas relaciones se ha explotado durante mucho tiempo para tratar enfermedades similares con los mismos tratamientos. Sin embargo, la mayoría de los enfoques tradicionales para descubrir estas relaciones han dependido de medidas subjetivas, como la similitud en los síntomas, sin obtener el conocimiento completo disponible a través del enriquecimiento.

\hfill

\subsection{Materiales}

En esta sección serán explicados los recursos web y las librerías de R utilizadas para los experimentos

\subsubsection{Human Phenotype Ontology}

HPO (Human Phenotype Ontology)\cite{HPO_paper} es un vocabulario estandarizado de anomalías fenotípicas en enfermedades humanas que utiliza un fenotipado detallado/preciso para poder ser usado a nivel computacional.

\hfill

\subsubsection{String}

STRING \cite{String_paper} es una base de datos de interacciones proteína-proteína conocidas y previstas. Las interacciones incluyen asociaciones directas (físicas) e indirectas (funcionales); se derivan de la predicción computacional, de la transferencia de conocimiento entre organismos y de interacciones agregadas de otras bases de datos (primarias).

\subsubsection{igraph}

Los objetivos de igraph son proporcionar un conjunto de tipos de datos y funciones para la implementación sencilla de algoritmos gráficos, el manejo rápido de gráficos grandes y permitir la creación rápida de prototipos a través de alta lenguajes de nivel como R

\subsubsection{Linkcomm}

Las comunidades enlazadas (link communities) pueden mostrar estructuras anidadas y otros tipos de estructuras superpuestas en las redes, llegando a descubrir nodos clave en la formación de conexiones con múltiples comunidades. Linkcomm \cite{Linkcomm_paper} proporciona herramientas para generar, visualizar y analizar comunidades de enlaces en redes de tamaño y tipo arbitrario.

\subsubsection{ClusterProfiler}

clusterProfiler \cite{ClusterProfiler_paper} implementa métodos para analizar y visualizar perfiles funcionales de coordenadas genómicas, genes y grupos de genes.

\subsubsection{org.Hs.eg.db} 

Anotación de todo el genoma para humanos, basada principalmente en el mapeo utilizando identificadores de genes Entrez

\subsubsection{Biomart}

La librería de R Biomart es un paquete de R que proporciona una interfaz para varias colecciónes de bases de datos, tanto bases de datos genéticas como moleculares \cite{biomart}. En concreto hemos utilizado varias funciones para el correcto mapeo de id de genes

\subsubsection{Gene Ontology (GO)}

La base de conocimientos Gene Ontology (GO) \cite{GO_doi} es la fuente de información más grande del mundo sobre las funciones de los genes. Este conocimiento es tanto legible por humanos como por máquinas, y es la base para el análisis funcional de experimentos genéticos y de biología molecular a gran escala en la investigación biomédica.






\hfill




\newpage