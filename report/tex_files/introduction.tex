\section{Introducción}

La Discalculia es una discapacidad de aprendizaje específica que afecta la adquisición de habilidades aritméticas. Aunque la falta de  enseñanza, recursos y la baja inteligencia se han relacionado con la etiología de la discalculia, se sabe actualmente que esta discapacidad del aprendizaje es un trastorno cerebral con una predisposición genética familiar \cite{Molko2003}.

\hfill

Las investigaciones se limitaron originalmente a participantes adultos, pero ahora también hay un número creciente de estudios en niños. La investigación en niños con un desarrollo típico, indica que la red fronto-parietal está constantemente activa durante el procesamiento de números y la aritmética \cite{originDis}. Ambos muestran similitudes y diferencias con lo que se está observando en adultos. Los niños con discalculia muestran anomalías tanto funcionales como estructurales en esta red.

\hfill


Por otra parte, hay estudios \cite{Molko2003,Shalev2001} que muestran que una forma de este fenotipo está asociado genéticamente con anomalías tanto funcionales como estructurales del surco intraparietal derecho, llevando esta región un papel fundamental en el desarrollo de las habilidades aritméticas.

\hfill

Enfermedades como Alzheimer u otras similares están asociadas con demencias fronto-temporales \cite{Walterfang2014} que están también fuertemente relacionadas con un número significativo de fenotipos en la discalculia.

\hfill

El diagnóstico de discalculia generalmente \cite{TreatmentDis} se realiza después de una evaluación exhaustiva, que incluye una revisión del historial del individuo, el desempeño en pruebas estandarizadas y un examen clínico. La evaluación adicional puede incluir una evaluación psico-social para identificar cualquier condición u otros factores que puedan estar contribuyendo a las dificultades matemáticas del individuo.

\hfill

La investigaciones muestran que el tratamiento puede ser eficaz para mejorar el rendimiento matemático en personas con discalculia, con un efecto medio de 0,52 \cite{TreatmentDis} en todos los ensayos de intervención. Es importante que las personas con discalculia reciban el apoyo y las adaptaciones adecuadas para ayudarlos a tener éxito en sus actividades académicas y profesionales.

\hfill

El tratamiento para la discalculia debe adaptarse a las áreas problemáticas específicas del individuo y puede incluir intervenciones  especializadas, tecnología de asistencia y adaptaciones en entornos educativos. Es importante que el tratamiento se inicie temprano \cite{ManagementDis}, preferiblemente en los años de la escuela primaria, y que lo lleven a cabo especialistas capacitados en un entorno individual. Las condiciones comórbidas, como la dislexia, el trastorno por déficit de atención/hiperactividad y otros trastornos mentales, también deben abordarse en el tratamiento.

\hfill

\subsection{Información sobre los genes a estudiar}


A continuación se dará una breve información sobre los genes de mayor grado de interconexión que hemos encontrado al realizar una búsqueda en una herramienta de análisis fenotípico conocida como HPO \cite{HPO_paper} (en la sección \emph{materiales} se explicará en profundidad y se darán detalles de como se ha usado), un recurso que usaremos para los experimentos, pero utilizado a modo informativo para detectar posibles recursos bibliográficos sobre los genes implicados en nuestro fenotipo

\hfill

\textbf{SQSTM1\cite{SQSTM1}:} Sequestosoma-1; Receptor de autofagia que interactúa directamente tanto con la carga a degradar como con un modificador de autofagia de la familia MAP1 LC3. Puede regular la activación de NFKB1 por TNF-alfa, factor de crecimiento nervioso (NGF) e interleucina-1. Puede desempeñar un papel en la señalización posterior de titina/TTN en las células musculares.

\hfill

\textbf{FUS\cite{FUS}:}Proteína de unión a ARN FUS; Se une tanto al ADN monocatenario como al bicatenario y promueve la hibridación independiente de ATP de los ADN monocatenarios complementarios y la formación del bucle D en el ADN superhelicoidal de doble cadena. Puede desempeñar un papel en el mantenimiento de la integridad genómica; Pertenece a la familia RRM TET.

\hfill

\textbf{VCP\cite{VCP}:} ATPasa del retículo endoplásmico de transición; Necesario para la fragmentación de las pilas de Golgi durante la mitosis y para su reensamblaje después de la mitosis. Participa en la formación del retículo endoplásmico de transición (tER). La transferencia de membranas desde el retículo endoplásmico al aparato de Golgi se produce a través de vesículas de transición de 50-70 nm que se derivan de elementos de transición parcialmente rugosos y parcialmente lisos del retículo endoplásmico (tER). La formación de vesículas en el tER es un proceso dependiente de ATP. El complejo ternario que contiene UFD1, VCP y NPLOC4 se une a proteínas ubiquitinadas.

\hfill

\textbf{CHMP2B\cite{CHMP2B}:}Proteína corporal multivesicular cargada 2b; Probable componente central de la clasificación endosomal requerida para el complejo de transporte III (ESCRT-III) que está involucrado en la formación de cuerpos multivesiculares (MVB) y la clasificación de proteínas de carga endosomal en MVB. Los MVB contienen vesículas intraluminales (ILV) que se generan por invaginación y escisión de la membrana limitante del endosoma y, en su mayoría, se envían a los lisosomas, lo que permite la degradación de las proteínas de la membrana, como los receptores del factor de crecimiento estimulado, las enzimas lisosomales y los lípidos.

\hfill

\textbf{HNRNPA2B1\cite{HNRNPA2B1}:}Ribonucleoproteínas nucleares heterogéneas A2/B1; Ribonucleoproteína nuclear heterogénea (hnRNP) que se asocia con pre-ARNm nacientes y los empaqueta en partículas de hnRNP. La disposición de las partículas de hnRNP en el hnRNA naciente no es aleatoria y depende de la secuencia, y sirve para condensar y estabilizar las transcripciones y minimizar los enredos y los nudos. El empaque juega un papel en varios procesos, como la transcripción, el procesamiento de pre-ARNm, la exportación nuclear de ARN, la ubicación subcelular, la traducción de ARNm y la estabilidad de los ARNm maduros. Forma partículas hnRNP con al menos otras 20 hnRNP diferentes.


\subsection{Introducción al experimento}

A continuación se proceden a detallar los pasos seguidos, las herramientas utilizadas y los resultados obtenidos en nuestro análisis personal de este fenotipo.

\hfill

Nuestra principal función investigadora consistirá en encontrar información relevante sobre las relaciones de los genes anteriormente citados añadidos a un conjunto mas grande, y comprobar mediante documentación científica si dichos resultados guardan alguna característica destacable que pueda aportar información relevante sobre la discalculia.

\newpage